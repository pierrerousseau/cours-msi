\section*{Exercice}

La FFJDA (Fédération française de judo, jujitsu, kendo et disciplines associées) souhaite créer un classement annuel des judokas se basant sur les résultats obtenus lors de compétitions officielles. La première étape sera la modélisation du système d'information porte sur la gestion des matchs et du classement. Pour cela on se basera uniquement sur le recueil des besoins suivant : \\
\begin{itemize}
	\item La fédération homologue les clubs pouvant organiser une compétition officielle.
	\item Tous les licenciés sont inscrits de fait aux compétitions officielles mais ne sont pas obligés d'y participer.
	\item Chaque compétition est organisée sous forme de poule où tous les judokas présents se rencontrent une et une seule fois (dans chaque catégorie). Chaque victoire donne trois points, chaque défaite donne un point. 
	\item Le club organisateur transmet les résultat de la compétition à la fédération une fois celle-ci terminée.
	\item Le système de poule pouvant générer des égalités, les clubs sont libres de choisir la manière de définir le vainqueur d'une compétition (mini-tournoi dont les matchs ne comptent pas pour le classement, nombre de ippons, ...) et souhaitent donc conserver un classement leur compétition au fil des ans.
\end{itemize}

\subsubsection*{documents fournis}

\begin{itemize}
	\item Licence de judo (numéro de licence, nom, date de naissance, sexe, catégorie de poids, club, année de validité)
	\item Feuille de match (compétition, date, gagnant, perdant)
	\item Classement (noms, victoires, défaites, points)
	\item homologation (nom et adresse du club, année)
	\item Classment d'une compétition (noms)
\end{itemize}
