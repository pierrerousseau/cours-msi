\section*{Exercice}

Une société souhaite moderniser son système d'information. Pour cela, elle décide de commencer par modéliser la partie de celui-ci liée au travail de son service commercial. Une première étude a mis en évidence les besoins suivants : \\
\begin{itemize}
    \item Chaque commercial doit avoir accès à ses prospects (clients potentiels) à travers le système d'information de l'entreprise.
    \item Un prospect n'est géré que par un seul commercial à un instant donné, même si celui-ci peut changer au cours du temps. On ne cherche pas à conserver d'historique de gestion des commerciaux ayant participé à la prospection.
    \item Un prospect qui signe un contrat reste un prospect, mais on conserve son historique de contrats pour les prospections suivantes.
    \item Un contrat n'est géré que par un seul commercial.
    \item On veut pouvoir calculer l'ensemble des revenus générés par les contrats signés par un commercial donné sur une période de temps donnée.
    \item On veut pouvoir calculer l'ensemble des coûts (salaires et notes de frais) d'un commercial sur une période de temps donnée.
\end{itemize}

\subsubsection*{documents jugés utiles}
\begin{itemize}
    \item Fiche de prospection (nom du client, adresse du client, numéro de téléphone du client, courriel du client, date de début de prospection)
    \item Contrat (nom du client, domaine d'activité, adresse du client, numéro de téléphone du client, date de signature, produits vendus)
    \item Fiche produit (code produit, désignation)
    \item Note de frais (date, désignation, coût, validée ou refusée)
    \item Fiche de paye (nom du commercial, date de naissance du commercial, adresse du commercial, salaire versé)
\end{itemize}
