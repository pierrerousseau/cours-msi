\section*{Exercice}

Le réseau de bibliothèques municipale de la ville de Nanterre souhaite revoir son système d'information. La première étape va consister à modéliser la gestion des prêts des livres de ces bibliothèques. Pour cela on se basera uniquement sur le recueil des besoins suivant : \\
\begin{itemize}
	\item Toutes les bibliothèques municipales sont accessibles aux adhérents. On devient adhérent en se présentant à une bibliothèque, une carte d'adhérent est remise immédiatement.
	\item Un adhérent ne peut emprunter que trois ouvrages au maximum et doit les rendre dans les trois semaines suivant l'emprunt. Tout manquement à cette règle entraîne l’envoi d'un courrier. Il peut emprunter et rendre un ouvrage dans n'importe quelle bibliothèque.
	\item Un livre non rendu après six semaines doit être remboursé, un courrier est envoyé. Les ouvrages endommagés ne sont pas acceptés en retour de prêt et doivent donc être remboursés au bout de six semaines. Si l'ouvrage n'est pas remboursé, l'adhérent est radié et ne peut plus s'inscrire.
	\item On veut pouvoir à tout moment savoir où se trouve un ouvrage en faisant une recherche sur son titre, son éditeur ou ses auteurs.
\end{itemize}

\subsubsection*{documents fournis}
\begin{itemize}
	\item Carte d'adhérent (numéro d'adhérent, nom, adresse, date et lieu de naissance, année de validité)
	\item Livre (code d'identification, titre, éditeur, prix)
	\item Courrier de rappel (titre du livre, date et lieu d'emprunt, numéro et nom d'adhérent)
	\item Courrier de remboursement (titre et prix du livre, date et lieu d'emprunt, numéro et nom d'adhérent)
	\item Courrier de radiation (titre et prix du livre, date et lieu d'emprunt, numéro et nom d'adhérent)
\end{itemize}
