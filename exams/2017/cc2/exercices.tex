\section*{Exercice}
Une association faisant la promotion du vélo permet à ses adhérents d'emprunter gratuitement des vélos.
L'association souhaite informatiser la gestion de ces vélos et de leurs emprunts.

\subsubsection*{les besoins et précisions suivantes vous sont communiqués :}
\begin{itemize}
    \item Les adhérents s’inscrivent une fois pour toute et conservent le même tarif de cotisation indéfiniment.
    \item Les adhérents doivent payer leur cotisation chaque année, sauf s'ils prêtent un vélo à l'association pour l'année, dans ce cas, ils n'ont pas de cotisation à payer.
    \item Un adhérent peut emprunter un vélo pour une période déterminée, à l'avance ou non, s'il est à jour de cotisation.
    \item Le trésorier tient au courant le gestionnaire des vélos des cotisations à jour.
    \item Quand un vélo est rendu, il est vérifié. S'il est cassé ou abimé il est envoyé en réparation, le réparateur préviens toujours de la date où il rendra le vélo.
\end{itemize}

\subsubsection*{les documents et données jugés utiles}
\begin{itemize}
    \item fiche vélo (numéro, type)
    \item fiche adhérent (année, numéro, nom, age, tarif cotisation, cotisation payée)
    \item calendrier (dates d'emprunts des vélos, dates de retour des vélos en réparation)
    \item fiche réparateur (nom, téléphone, adresse)
\end{itemize}

%\subsubsection*{notes}
%\begin{itemize}
%    \item 
%\end{itemize}
