\section*{Exercice}
Un meunier (fabricant de farine) désire revoir la partie de son système d'information gérant les commandes de sac de farine de ses clients.

\subsubsection*{les besoins et précisions suivantes vous sont communiqués :}
\begin{itemize}
    \item Le meunier désire connaître l'état des commandes en cours (commande livrée entièrement, factures payées)
    \item Pour chaque commande, une ou plusieurs livraisons sont préparées par le meunier
    \item À chaque livraison correspond une et une seule facture
    \item Chaque matin, le meunier vérifie les commandes validées et les prépare
    \item Pour chaque livraison validée par le client, le trésorier est prévenu automatiquement et une facture est envoyée
    \item Un client peut posséder plusieurs boulangeries à livrer, mais une seule adresse de facturation
\end{itemize}

\subsubsection*{les documents et données jugés utiles}
\begin{itemize}
    \item Bon de commande (adresses de livraison et adresse de facturation, type et nombre de sacs commandés)
    \item Validation de commande (numéro de référence de commande)
    \item Bon de livraison (numéro de référence de livraison, nom et adresse de livraison, type et nombre de sacs livrés, numéro de référence de commande)
    \item Validation de livraison (numéro de référence de livraison)
    \item Facture (nom et adresse de facturation, numéro de référence de livraison)
\end{itemize}

%\subsubsection*{notes}
%\begin{itemize}
%    \item 
%\end{itemize}
