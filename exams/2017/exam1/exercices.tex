\section*{Exercice}
Une usine de fabrication de petits jouets robots désire revoir la partie de son système d'information gérant les nomenclatures de fabrications des robots.

\subsubsection*{les besoins et précisions suivantes vous sont communiqués :}
\begin{itemize}
    \item Les ingénieurs de l'usine travaillent sur les plans des nouveaux robots et les fournissent au chef de département.
    \item Celui-ci les valide et fournis les plans à la cellule de fabrication, où travaillent des ouvriers fabricant les robots, et au gestionnaire du stock qui commande les pièces nécessaires.
    \item Les ingénieurs fournissent également directement les plans d'éléments généraux à la cellule de fabrication.
    \item Les pièces et les éléments sont ensuite fournis par le gestionnaire du stock à la cellule de fabrication selon les besoins.
    \item Une fois fabriqués, les robots et les éléments sont placés dans le stock.
    \item Chaque mois, le chef de département affecte à des commerciaux prestataires des objectifs de ventes.
\end{itemize}

\subsubsection*{les documents et données jugés utiles :}
\begin{itemize}
    \item plan de robot (nom, nombres et éléments utilisées)
    \item plan d'élément (numéro, éléments et pièces utilisées)
    \item fiche pièce (désignation)
    \item fiche objectif (nom du commercial, nom du robot, nombre de ventes, date limite)
\end{itemize}

\subsubsection*{notes :}
\begin{itemize}
    \item On ne modélise ni la gestion des commandes, ni la gestion des ventes.
    \item Les objectifs de ventes ne sont pas historisés.
    \item Une pièce peut servir dans différents éléments et les éléments dans différents éléments ou robots.
\end{itemize}
