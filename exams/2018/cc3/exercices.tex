\section*{Exercice}
Un gérant d'hôtel cherche à mieux gérer son système de promotion. Aider le à modéliser cette partie de son système d'information.

\subsubsection*{les besoins et précisions suivantes vous sont communiqués :}
\begin{itemize}
    \item S'ils le désirent les clients peuvent s'enregistrer auprès de la réception de l'hôtel pour bénéficier de réductions. Ils reçoivent alors une copie de leur fiche client.
    \item Pour chaque chambre, une ou plusieurs promotions sont disponibles (en fonction du nombre de fois où cette chambre a été réservée). Ces promotions sont décidées au début de l'année par le directeur de l'hôtel.
    \item Lors de la location, un client enregistré peut accepter une promotion ou la refuser et attendre d'en avoir une meilleure.
    \item Lorsqu'un client accepte une promotion, ses locations précédant cette date pour cette chambre ne comptent plus (mais on les conserve pour historique).
\end{itemize}

\subsubsection*{les documents et données jugés utiles :}
\begin{itemize}
    \item Fiche client (nom, numéro de téléphone, nombre de locations par chambre, dates des demandes de promotion par chambre)
    \item Fiche Location (chambre louée, prix, date de location)
    \item Fiche Promotion (chambre associée, nombre de locations pour obtention, réduction, année)
\end{itemize}

\subsubsection*{notes :}
\begin{itemize}
    \item On se base sur les dates d'acceptation de promotion et de location pour compter le nombre de locations en cours
    \item On considère qu'un client ne peut pas obtenir deux fois la même promotion.
\end{itemize}
