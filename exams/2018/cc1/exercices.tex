\section*{Exercice}
Suite à des problèmes de décompte des votes, le délégué d'une classe de MIAGE 
souhaite mettre en place un système de vote pour prendre des décisions 
collectives sans contestation.

\subsubsection*{les besoins et précisions suivantes vous sont communiqués :}
\begin{itemize}
    \item Chaque élève de la classe désirant voter doit demander une carte de votant
    \item Pour commencer le vote, le délégué choisit un ou plusieurs élèves pour enregistrer les votes
    \item Chaque choix est conservé, chaque élève ne pouvant faire qu'un choix par vote
    \item À la fin du vote, le résultat est communiqué au demandeur du vote
    \item Un demandeur de vote peut consulter l'état de ses votes à tout instant
\end{itemize}

\subsubsection*{les documents et données jugés utiles}
\begin{itemize}
    \item Carte de votant (nom, prénom, identifiant étudiant, votes)
    \item Liste des choix (intitulé, destinataire (nom, prénom, fonction), nombre de voix par choix)
\end{itemize}

\subsubsection*{notes}
\begin{itemize}
    \item La fonction d'un demandeur est par exemple : professeur, élève, administratif, ...
    \item On conserve les élèves désignés pour enregistrer les votes, mais pas quel élève à enregistré quel vote
\end{itemize}
