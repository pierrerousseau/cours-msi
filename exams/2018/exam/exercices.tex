\section*{Exercice}
Un directeur de médiathèque souhaite changer l'organisation de ses rayons de disques. Modélisez la partie du systèmes d'information correspondant à ses besoins.


\subsubsection*{les besoins et précisions suivantes vous sont communiqués :}
\begin{itemize}
    \item Le directeur veut un rayon avec les meilleurs disques, toutes catégories confondues.
    \item Pour savoir quels sont les meilleurs disques, il désire que les adhérents puissent noter les titres des disques.
    \item Il veut également permettre aux adhérents de qualifier les titres pour faciliter la recherche de disques.
    \item Les notes et les qualificatifs sont donnés à l'accueil lors du retour d'un emprunt.
    \item Chaque semaine, le directeur calcule la note moyenne de chaque disque. Le chef de rayon place ensuite dans le rayon des meilleurs disques ceux ayant les meilleures notes.
    \item Le directeur veut pouvoir gérer tout cela depuis sa tablette car il passe beaucoup de temps hors de la médiathèque avec ses amis artistes.
\end{itemize}

\subsubsection*{les documents et données jugés utiles :}
\begin{itemize}
    \item carte d'adhérent (nom, prénom, notes)
    \item fiche disque (nom, titres (nom, artistes, qualificatifs, note), note moyenne)
    \item étiquette rayon (nom)
\end{itemize}

\subsubsection*{notes :}
\begin{itemize}
    \item l'adhésion et l'emprunt des disques sont gérés à l'accueil
    \item une seule note par adhérent est acceptée pour un titre 
    \item plusieurs qualificatifs par adhérent sont acceptés pour un titre
    \item on ne modélise pas la recherche de titre par les adhérents
\end{itemize}
