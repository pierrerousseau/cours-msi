\section*{Exercice}
Une université cherche à modéliser la gestion des inscriptions à ses différentes bibliothèques. 
Aider la à modéliser la partie de son système d'information correspondante.

\subsubsection*{les besoins et précisions suivantes vous sont communiqués :}
\begin{itemize}
    \item Un étudiant, lorsqu'il s'inscrit à l'université reçoit sa carte d'étudiant et est automatiquement inscrit dans certaines bibliothèques en fonction du diplome préparé (matières et année). 
    \item Un étudiant peut se rendre dans une bibliothèque où il n'est pas inscrit et demander de s'y inscrire. La demande peut être acceptée ou refusée.
    \item À la fin de l'année scolaire, les directeurs de bibliothèques se réunissent. Ils veulent connaître le nombre de demandes d'inscriptions aux différentes bibliothèques pour fournir au service d'inscription de l'université la liste des inscriptions automatiques.
\end{itemize}

\subsubsection*{les documents et données jugés utiles :}
\begin{itemize}
    \item carte d'étudiant (année en cours, nom, niveau préparé (L1, ..., M2, Recherche), matière (informatique, histoire, ...), inscriptions aux bibliothèque (validées et refusées))
    \item list des inscriptions automatiques à une bibliothèque (identifiant bibliothèque, niveaux (L1, ..., M2, Recherche) et matières donnant droit à l'inscription automatique (informatique, histoire, ...))
\end{itemize}

\subsubsection*{notes :}
\begin{itemize}
    \item un étudiant reçoit une nouvelle carte d'étudiant chaque année.
\end{itemize}
