\section*{Exercice}
Le gérant d'un espace de co-working cherche à modéliser son système d'information. Un espace de co-working est un lieu où l'on peut venir travailler et rencontrer des professionnels de différents corps de métiers. On loue un espace pendant une journée ou plus dans un lieu commun à d'autres clients de l'espace de co-working. On peut également y louer un bureau privé, pour organiser une réunion ou autre. La location se fait à la demi-journée et comprend la connexion internet ainsi que l'accès à un espace "repos" où des boissons, des pâtisseries et des fruits sont offerts.
Pour commencer sa modélisation, le gérant veut s'occuper de la gestion des places disponibles et ne pas louer plus de bureaux qu'il n'en a.


\subsubsection*{les précisions suivantes vous sont communiqués}
\begin{itemize}
    \item Chaque bureau est dans une salle.
    \item On peut louer un bureau ou une salle.
    \item On ne doit pas louer un bureau à plusieurs personnes.
    \item Les clients s'inscrivent auprès du bureau d'inscription et reçoive une carte.
    \item N'importe qui peut faire une demande de planning d'occupation des bureaux au gestionnaire de salles.
    \item Une demande de location doit être faite par un client et être validée par le gestionnaire de salles.
    \item À la fin de chaque mois, le gestionnaire de salles envoie un bilan d'information au gérant avec le nombre de bureaux loués dans le mois et le chiffre d'affaire du mois.
\end{itemize}

\subsubsection*{les documents et données jugés utiles}
\begin{itemize}
    \item Fiche descriptive d'une salle (nom de salle, nombre de bureaux).
    \item Fiche descriptive d'un bureau (numéro, salle, prix).
    \item Carte client (numéro, nom client).
    \item Bilan mensuel (nombre de bureaux loués, CA)
\end{itemize}

%\subsubsection*{notes}
%\begin{itemize}
%    \item 
%\end{itemize}
