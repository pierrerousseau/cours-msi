\section*{Exercice}
Une petite municipalité souhaite améliorer son service de bus. Elle a remarqué qu'à certaines heures, plusieurs de ses bus ne prenaient en charge que quelques personnes et tournaient parfois à vide. Elle a donc décidé de supprimer ces tournées. Pour ne pas léser ces usagers, ceux-ci peuvent désormais appeler un standard depuis l'arrêt de bus et leur envoie un un mini-van pour les prendre en charge (les mini-vans passent uniquement aux stations où un usager a appelé). Si personne n'appelle avant l'horaire de passage, aucun service n'est effectué.

\subsubsection*{les précisions suivantes vous sont communiqués}
\begin{itemize}
    \item Aujourd'hui, sur chaque ligne de bus, une tournée est réalisée chaque heure.
    \item Chaque année, les tournées sont redéfinies par le chef de service pour des raisons que nous ne prendrons pas en compte dans notre modélisation. Il peut soit conserver une tournée, soit la supprimer, soit en créer de nouvelles.
    \item À la fin de chaque mois, après de savants calculs, il est décidé des tournées réalisées en bus et de celles pour lesquelles les usagers peuvent appeler un mini-van. Le chef de service informe les chauffeurs du type de véhicule à utiliser pour chaque tournée.
    \item Lors d'un appel à une borne, le standard prévient le chauffeur du mini-van s'il n'est pas déjà passé à cette station sinon le standard demande à l'usager d'attendre la prochaine tournée.
\end{itemize}

\subsubsection*{les documents et données jugés utiles}
\begin{itemize}
    \item Ticket composté (date, nom de station)
    \item Fiche tournée (noms des stations desservies et heures de passage aux stations, heure de départ, année de début, année de fin, véhicules)
    \item Fiche véhicule (numéro de véhicule, type de véhicule (bus, mini-van), consommation horaire (en litre par heure))
\end{itemize}

%\subsubsection*{notes}
%\begin{itemize}
%    \item 
%\end{itemize}
